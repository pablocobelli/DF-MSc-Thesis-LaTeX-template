% PAQUETES NECESARIOS
\usepackage[utf8]{inputenc}                 % para poder esribir con acentos
\usepackage[T1]{fontenc}                    % encoding T1 para fonts
\usepackage[english,spanish]{babel}         % multilenguaje
\usepackage{graphicx}                       % inclusion de figuras 
\usepackage{amsthm, amsmath, amssymb}       % fonts y environments para math
\usepackage{setspace}\onehalfspacing        % espaciado entre lineas
\usepackage[loose,nice]{units}              % units in upright fractions
\usepackage{DF-MSc-titlepage}               % titlepage al estilo df.uba.ar
\usepackage{indentfirst}                    % indentar el inicio de seccion
\usepackage{lipsum}                         % para generar texto generico
\usepackage{aas_macros}                     % macros para nombre de journals
\usepackage{hyperref}                       % para vinculos en el pdf
\usepackage{bookmark}                       % para que genere bookmarks en el pdf
\usepackage{fancyhdr}                       % para los headers & footers
\usepackage{emptypage}                      % saca headers and footers de paginas en blanco 
\usepackage[margin=1in]{geometry}           % geometria de la pagina 

% CUSTOMIZACIONES PROPIAS
\graphicspath{{./figures/}}                 % define el directorio de figuras
\usepackage[Sonny]{fncychap}                % para definir estilos de capitulos
\renewcommand{\vec}[1]{\mathbf{#1}} 	    % vectores como bold

\geometry{bindingoffset=1cm}                % espacio en el borde interno para el encuadernado
\geometry{textwidth=390pt}                  % cuerpo del texto fijo en 390pt
\addto\captionsspanish{\renewcommand{\listtablename}{Índice de tablas}}
